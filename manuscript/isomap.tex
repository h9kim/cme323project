\documentclass{article} % For LaTeX2e
\usepackage{nips13submit_e,times}
\usepackage{hyperref}
\usepackage{url}
\usepackage{amssymb, amsmath}
\usepackage{epsfig}
\usepackage{array}
\usepackage{ifthen}
\usepackage{color}
\usepackage{fancyhdr}
\usepackage{graphicx}
%\documentstyle[nips13submit_09,times,art10]{article} % For LaTeX 2.09


\title{Large-scale approximate ISOMAP}


\author{
Charles Y.~Zheng and Jinshu Wang\\
Department of Statistics\\
Stanford University\\
Stanford, CA 94305 \\
\texttt{\{snarles, jinshuw\}@stanford.edu} \\
\and
\textbf{Arzav ~Jain} \\
Department of Computer Science\\
Stanford University\\
Stanford, CA 94305 \\
\texttt{arzavj@stanford.edu} \\
}

% The \author macro works with any number of authors. There are two commands
% used to separate the names and addresses of multiple authors: \And and \AND.
%
% Using \And between authors leaves it to \LaTeX{} to determine where to break
% the lines. Using \AND forces a linebreak at that point. So, if \LaTeX{}
% puts 3 of 4 authors names on the first line, and the last on the second
% line, try using \AND instead of \And before the third author name.

\newcommand{\fix}{\marginpar{FIX}}
\newcommand{\new}{\marginpar{NEW}}

\nipsfinalcopy % Uncomment for camera-ready version

\begin{document}


\maketitle

\begin{abstract}
  Isometric feature mapping (ISOMAP) is an unsupervised method for
  nonlinear dimensionality reduction (Tenenbaum et al. 2000), intended
  for exploratory data analysis or feature extraction of complex data
  with hidden low-dimensional structure, such as text, images, or
  neurological data.  ISOMAP requires three computationally intensive
  subroutines: formation of a nearest-neighbors graph, computation of
  all-pairs shortest-paths for the nearest-neighbors graph and
  computation of the largest eigenvectors of the centered all-pairs
  distance matrix.  Traditionally, computation of the
  all-pairs-distance matrix is $O(n^3)$, and computation of the top
  eigenvectors is $O(n^2d)$, hence application of single-core ISOMAP
  scales poorly for large datasets.  Here we propose two approaches to
  approximating ISOMAP within a distributed framework.  The first is
  to compute a random subset of the columns of the all-pairs-distance
  matrix, thus approximating the eigenvectors of the full distance
  matrix by the left singular vectors of the submatrix.  This results
  in substantial computational savings because single-source shortest
  paths algorithms can be used to compute the submatrix rather than
  all-pairs algorithms, and also because the SVD of the submatrix is
  much cheaper to compute than the eigenvectors of the full matrix.
  Meanwhile, the approximation error can be bounded using results on
  column-sampling SVD (Frieze 2004).  Our second approach is to
  compute an approximation of the full all-pairs shortest-paths using
  a blocked version of Floyd-Warhsall algorithm for all-pairs shortest
  paths.  Our blocked Floyd-Warshall sacrifices accuracy for improved
  latency; assuming that the data has been pre-shuffled, we show that
  its error is bounded given appropriate block sizes for small
  diameter graphs (such as the nearest-neighbor graphs required for
  ISOMAP).  We implement these approaches in Apache Spark and
  demonstrate their application on synthetically generated data
  as well application in fMRI data.
\end{abstract}

\section{Introduction}

Isometric feature mapping (ISOMAP) is an unsupervised method for
nonlinear dimensionality reduction \cite{Tenenbaum2000}, intended
for exploratory data analysis or feature extraction of complex data
with hidden low-dimensional structure, such as text, images, or
neurological data.  ISOMAP requires three computationally intensive
subroutines: formation of a nearest-neighbors graph, computation of
all-pairs shortest-paths for the nearest-neighbors graph and
computation of the largest eigenvectors of the centered all-pairs
distance matrix.  Traditionally, computation of the all-pairs-distance
matrix is $O(n^3)$, and computation of the top eigenvectors is
$O(n^2d)$, hence application of single-core ISOMAP scales poorly for
large datasets.

While a large body of work exists for implementing all-pairs
shortest-paths and eigendecomposition in supercomputers, our work is
the first to leverage specific properties of the ISOMAP dimension
reduction problem to efficiently compute an approximation to the
ISOMAP coordinates.  A secondary difference between our work and the
previous literatue is that we are motivated specifically by the Spark
framework.  While our results could be applicable to distributed
systems in general and even to single-core systems, our algorithms
should be especially well-suited for an advanced cluster computing
framework like Apache Spark \cite{Zaharia2010}.

Here we propose two approaches to approximating ISOMAP within a
distributed framework.  The first is to compute a random subset of the
columns of the all-pairs-distance matrix, thus approximating the
eigenvectors of the full distance matrix by the left singular vectors
of the submatrix.  This results in substantial computational savings
because single-source shortest paths algorithms can be used to compute
the submatrix rather than all-pairs algorithms, and also because the
SVD of the submatrix is much cheaper to compute than the eigenvectors
of the full matrix.  Meanwhile, the approximation error can be bounded
using results on column-sampling SVD\cite{Frieze2004}.

A second approach could be start by computing the full all-pairs
shortest-paths matrix using a distributed algorithm.  There exists a
large body of work for all-pairs shortest-paths in the context of
supercomputing, of these, the most suitable for implementation in the
\cite{Kumar1991} Our second approach is to compute an approximation of
the full all-pairs shortest-paths using a blocked version of
Floyd-Warhsall algorithm for all-pairs shortest paths.  Our blocked
Floyd-Warshall sacrifices accuracy for improved latency; assuming that
the data has been pre-shuffled, we show that its error is bounded
given appropriate block sizes for small diameter graphs (such as the
nearest-neighbor graphs required for ISOMAP).  We rely on classical
results on the generalized birthday problem \cite{kolchin1978random}
for these error bounds.

\small{
\bibliographystyle{abbrv}
\bibliography{isomap}
}



\end{document}
