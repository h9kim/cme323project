\documentclass{article} % For LaTeX2e
\usepackage{nips13submit_e,times}
\usepackage{hyperref}
\usepackage{url}
\usepackage{amssymb, amsmath}
\usepackage{epsfig}
\usepackage{array}
\usepackage{ifthen}
\usepackage{color}
\usepackage{fancyhdr}
\usepackage{graphicx}
\usepackage{algorithm}
\usepackage{algpseudocode}
\usepackage{mdframed}
%\documentstyle[nips13submit_09,times,art10]{article} % For LaTeX 2.09


\title{Approximate All-Pairs Shortest Paths}


\author{
Charles Y.~Zheng and Jinshu Wang\\
Department of Statistics\\
Stanford University\\
Stanford, CA 94305 \\
\texttt{\{snarles, jinshuw\}@stanford.edu} \\
\and
\textbf{Arzav ~Jain} \\
Department of Computer Science\\
Stanford University\\
Stanford, CA 94305 \\
\texttt{arzavj@stanford.edu} \\
}

% The \author macro works with any number of authors. There are two commands
% used to separate the names and addresses of multiple authors: \And and \AND.
%
% Using \And between authors leaves it to \LaTeX{} to determine where to break
% the lines. Using \AND forces a linebreak at that point. So, if \LaTeX{}
% puts 3 of 4 authors names on the first line, and the last on the second
% line, try using \AND instead of \And before the third author name.

\newcommand{\fix}{\marginpar{FIX}}
\newcommand{\new}{\marginpar{NEW}}

\nipsfinalcopy % Uncomment for camera-ready version

\begin{document}


\maketitle

\begin{abstract}
  This document describes in detail our approximate algorithm for
  all-pairs-shortest-paths, Block Floyd-Warshall.  It will be later
  incorporated into the Appendix for the paper {\tt isomap.pdf}.
\end{abstract}

\section{Introduction}

\subsection{Definitions}

Let $G = (V, E)$ be a weighted directed graph with weights $w_{i,j}$ for edges $(i \to j) \in E$.
Define the adjacency matrix $A$ as a square matrix with dimension $n = |V|$, and entries
\[
A_{ij}
\begin{cases}
w_{i, j} &\text{ if } (i \to j) \in E\\
0 &\text{ if } i = j\\
\infty &\text{ if } (i \to j) \notin E
\end{cases}
\]
Further assume that $A$ is strongly connected; that is, for all $i, j
\in A$, there exists a \emph{path} $(v_0 \to v_1 \to \hdots \to v_l)$
with $v_0 = i$, $v_l = j$, $l \geq 0$ such that for all $k \in 0,
\hdots, l$, $(v_k \to v_{k+1}) \in E$.  For any given path $p = (v_0
\to \hdots \to v_l)$, define the weight of the path $w(p)$ as the sum
of the edges
\[
w(p) = \sum_{i=1}^l e_{v_i, v_{i+1}}
\]
and define the \emph{length} of the path $\ell(p)$ as the number of
edges in the path, $\ell(p) = l$.  For $i, j \in V$ let
$\mathcal{P}_{i,j}$ denote the set of all directed paths from $i$ to
$j$, and $\tilde{\mathcal{P}}_{i, j}$ denote the set of \emph{shortest
  directed paths} from $i$ to $j$, defined by
\[
\tilde{\mathcal{P}}_{i,j} = \{\tilde{p} \in \mathcal{P}_{i,j}: w(\tilde{p}) = \min_{p \in \mathcal{P}_{i,j}} w(p)\}
\]
Let $\ell_{i,j}$ be the \emph{minimum length} of the shortest directed
path in $\mathcal{P}_{i,j}$,
\[
\ell_{i,j} = \min_{p \in \tilde{\mathcal{P}}_{i,j}} \ell(p)
\]
Finally, let $\ell_G$ be the \emph{shortest-path diameter} of the graph $G$, defined as
\[
\ell_G = \max_{i, j \in V} \ell_{i, j}
\]


Given an $n \times m$ matrix $A$ and an $n \times m$ matrix $B$, define the entrywise minimum $C = \min(A, B)$ by
\[
C_{ij} = \min(A_{ij}, B_{ij})
\]
Meanwhile, given an $n \times k$ matrix $A$ and a $k \times m$ matrix $B$,
define the \emph{min-plus} product $C = A \otimes B$ by
\[
C_{i,j} = \min_{l = 1}^k A_{il} + B_{lj}
\]
for $i = 1,\hdots, n$ and $j = 1,\hdots, m$.
For a square matrix $A$, let $A^k$ denote the min-plus $k$th power of $A$,
\[
A^k = \underbrace{A \otimes A \otimes \cdots \otimes A}_{\text{$k$ times}}
\]

\subsection{All-pairs shortest paths}

We consider the problem of \emph{all-pairs shortest-paths} (APSP) as the
following: given an $n \times n$ adjacency matrix $A$ describing graph
$(G,V)$, compute the $n \times n$ matrix $S$ if all shortest-path distances, with
\[
S_{ij} = \min_{p \in \mathcal{P}_{i,j}} w(p)
\]

Define the $k$-shortest path matrix $S^{(k)}$ by
\[
S^{(k)}_{ij} = \begin{cases} \min_{p \in \mathcal{P}_{i, j}: \ell(p) \leq k} w(p) &\text{ if }\min_{p \in \mathcal{P}_{i,j}} \ell(p) \leq k\\
\infty & \text{otherwise}
\end{cases}
\]
Note that $A = S^{(1)}$; it is easily verified that $A^k = S^{(k)}$
for all $k \geq 1$.  Furthermore, we have $S^{(\ell_G)} = S$ since for
any pair $i, j \in V$ there exists a shortest path with length at most
$\ell_G$.

This motivates the iterative squaring algorithm for shortest pairs.

\begin{algorithm}[H]
\caption{Iterative Squaring for APSP}
\begin{algorithmic}
\Function{IterSquare}{Adjacency matrix $A$}
  \State $S \leftarrow A$
  \For{$k = 1,\hdots, \lceil \log_w(\ell_G) \rceil $}
    \State $S \leftarrow S^2$
  \EndFor
  \State Return $S$
\EndFunction
\end{algorithmic}
\end{algorithm}

On a single machine, the cost of each squaring operation is $O(n^3)$.
Hence the computational cost of iterative squaring is $O(\log(\ell_G) n^3)$.
Since $\ell_G$ is at most $n$, the worst-case cost is therefore $O(\log(n) n^3)$.
The Floyd-Warshall algorithm improves substantially on this cost:

\begin{algorithm}[H]
\caption{Floyd-Warshall algorithm for APSP}
\begin{algorithmic}
\Function{FloydWarshall}{Adjacency matrix $A$}
  \State $S \leftarrow A$
  \For{$k = 1,\hdots, n $}
    \State Let $S_{\cdot, k}$ denote the $k$th column and $S_{k, \cdot}$ denote the $k$th row of $S$
    \State $S \leftarrow \min(S, S_{\cdot, k} \otimes S_{k, \cdot})$
  \EndFor
  \State Return $S$
\EndFunction
\end{algorithmic}
\end{algorithm}

The cost of computing $\min(S, S_{\cdot, k} \otimes S_{k, \cdot})$ in
each iteration is $O(n^2)$.  Since there are $n$ iterations, the cost
is therefore $O(n^3)$, which is strictly better than iterated squaring
regardless of the shortest-path-diameter of the graph.  However, we
will see that in a distributed setting, Floyd-Warshall is no longer
strictly better.

\subsection{Distributed setting}

Consider a network of $p$ worker nodes arranged in an $\sqrt{p} \times
\sqrt{p}$ grid.  We denote the entire grid by $\Lambda$ and a
particular worker by $\Lambda[i,j]$.  For any $n \times n$ square
matrix $A$, we say that $A$ is stored on the grid if $\Lambda[i,j]$
stores corresponding $n/\sqrt{p} \times n/\sqrt{p}$ block of $A$.  The
crucial property of this block structure is that for any pair of
matrices $A$, $B$, stored on the grid, then if the number of rows of
$A$ match the number of rows of $B$, then the number of rows of each
block of $A$ matches the number of rows of each corresponding local
block of $B$, and that the analagous property holds if the number of
columns of $A$ matches the number of columns of $B$.

Hence our setup resembles the computing grids studied in the
supercomputing literature\cite{Kumar1991} except that we make
different assumptions about communication costs, which are more
appropriate to distributed computing ``in the cloud''.  Given our
motivation of developing algorithms for cloud computing, our main
objective is to measure the total \emph{waiting time} needed for the
entire calculation to complete, and secondarily the total
\emph{expense} as measured by number of workers times waiting time.
These measures of cost are the most appropriate for cloud computing,
where waiting time may be a crucial factor (e.g. for streaming
applications) and where financial cost corresponds to worker-hours.

In particular, we separate the cost of communication and computation,
so that sending messages is assumed to cost zero CPU; however, we only
allow one message to be sent or recieved at any time, and a machine
that is sending is not open for recieving and vice versa.  However,
mutual disjoint pairs of workers can communicate simultaneously.
Also, we assume a homogenous cost of communication between any pair of
workers.  Let a message $M$ consist of $w$ words.  The time $T(M)$ it
takes for a worker $\Lambda[i,j]$ to send a message $M$ to another
worker $\Lambda[k,l]$ consisting of $w$ words is determined by the
latency of the network $\kappa_L$ and the inverse transmission rate
$\kappa_T$ and is given by
\[
T(M) = \kappa_L + \kappa_T w
\]
Following these assumptions, the cost for a single worker to broadcast
a message to $q$ other workers is given by
\[
T_B(M) \approx \log(q) (\kappa_L + \kappa_T w) 
\]
since approximately $\log(q)$ rounds of one-to-one transmission are
needed.  If multiple messages $M$ are broadcasted simultaneously
(possible from the same source), the one-to-one communication rounds
can be arranged in a way so that the waiting time for the entire
process is only a constant larger than the time for the $\max_M
T_B(M)$.  Let therefore make the simplifiying assumption that the cost
for the simultaneous broadcast of $b$ messages, all of size $w$, and
each to $q$ recipients is
\[
T_B(M_1,\hdots, M_b) \approx \log(q)(\kappa_L + \kappa_T w) + \kappa_S b
\]
where $\kappa_S$ is a constant describing the additional time per
message.

We take the computational cost for a single worker to complete one
operation to be $\kappa_C$.  On a single core, we take the time of
entrywise minimum to be $\kappa_C nm$ and the time of min-plus
multiplication to be $2\kappa_C nmk$, for input matrices of the
dimensions as described in the respective definitions.

We define the following algorithms for distributed entrywise minimum
and distributed min-plus multiplication.

\begin{algorithm}[H]
\caption{Distributed Entrywise Minimum}
\begin{algorithmic}
\Function{DistributedMin}{$n\times m$ matrices $A$, $B$}
  \State Let $A[i, j]$ denote the block of $A$ owned by worker $\Lambda[i, j]$, and similarly define $B[i, j]$
  \For{$i, j = 1,\hdots, \sqrt{p} $ in parallel}
    \State Define $C[i, j] = \min(A[i, j], B[i, j])$
  \EndFor
  \State Result $C$ is stored on grid $\Lambda$
\EndFunction
\end{algorithmic}
\end{algorithm}

\begin{algorithm}[H]
\caption{Distributed Min-Plus multiplication}
\begin{algorithmic}
\Function{DistributedMPM}{$n\times k$ matrix $A$, $k \times m$ matrix $B$}
  \State Let $A[i, \cdot]$ denote the blocks of $A$ owned by workers $\Lambda[i, \cdot]$, and similarly define $B[\cdot, j]$
  \For{$i = 1,\hdots, \sqrt{p}$ and $j = 1,\hdots, \sqrt{p}$, asynchronously}
    \State Broadcast $A'_i = A[i, \cdot]$ to all workers $\Lambda[i, \cdot]$
    \State Broadcase $B'_j B[\cdot, j]$ to all workers $\Lambda[\cdot, j]$
  \EndFor
  \For{$i, j = 1,\hdots, \sqrt{p}$ in parallel}
    \State Set $C[i, j] \to A'_i \otimes B'_j$
  \EndFor
  \State Result $C$ is stored on grid $\Lambda$
\EndFunction
\end{algorithmic}
\end{algorithm}

Here the notation $A'_i$, $B'_j$ is intended to distinguish broadcasted copies from local matrices.
The algorithm {\sc DistributedMPM} only works if each block $A[i, \cdot]$ and $B[\cdot, j]$ fit in worker memory.
Otherwise, one has to split the job into smaller parts.

\begin{algorithm}[H]
\caption{Large-scale distributed Min-Plus multiplication}
\begin{algorithmic}
\Function{DistributedMPM2}{$n\times k$ matrix $A$, $k \times m$ matrix $B$}
  \State Let $A = [A_1,\hdots, A_q]$ where each $A_i$ fits in memory.
  \State Let $B^T = [b_1^T, \hdots, b_q^T]$ where each $b_i$ fits in memory.
  \State Intialize distributed $C = 0$ with dimension $n \times m$
  \For{$k = 1,\hdots q$}
    \State $C \leftarrow \text{\sc DistributedMin}(C, \text{\sc DistributedMPM}(A_k, b_k))$
  \EndFor
  \State Result $C$ is stored on grid $\Lambda$
\EndFunction
\end{algorithmic}
\end{algorithm}

We now state the cost of of these algorithms assuming that the inputs
are already in place on the network.
\begin{itemize}
\item {\sc DistributedMin} requires no communication, hence the cost
  is the cost of communication.  Each worker must complete a local
  entrywise min, hence the waiting time is $\kappa_C nm/p$.
\item {\sc DistributedMP} requires a simultaneous broadcast of
  $2\sqrt{p}$ messages, each message having size at most $k\max(n,m)/\sqrt{p}$ and $\sqrt{p}$ recipients.
  Afterwards, each worker must complete a local min-plus multiplication.
  Hence, the waiting time is bounded by
\[
\frac{1}{2}\log(p) (\kappa_L + \kappa_T k \max(n, m)/\sqrt{p}) + \kappa_C nmk/p
\] 
\end{itemize}
We will analyze the cost of {\sc DistributedMPM2} in the following analyses.

Henceforth we use {\sc DistributedMin}, {\sc DistributedMPM}, and {\sc
  DistributedMPM2} as building blocks for our distributed versions of
{\sc IterSquare} and {\sc FloydWarshall}.  Up to a change in notation,
our distributed Floyd-Warshall is essentially the same as described in
Kumar \cite{Kumar1991}.

\begin{algorithm}[H]
\caption{Distributed Iterative Squaring}
\begin{algorithmic}
\Function{D-IterSquare}{Adjacency matrix $A$}
  \State $S \leftarrow A$
  \For{$k = 1,\hdots, \lceil \log_w(\ell_G) \rceil $}
    \State $S \leftarrow \text{\sc DistributedMPM2}(S, S)$
  \EndFor
  \State Result $S$ stored on grid $\Lambda$
\EndFunction
\end{algorithmic}
\end{algorithm}

Let us assume that each worker can store at least $3n^2/p$ words in
RAM. It follows that $p$ chunks are needed in {\sc DistributedMPM2},
which therefore involves an alternating sequence of $p$ calls each of
{\sc DistributedMPM} and {\sc DistributedMin}, where each call can
only begin as soon as the previous call is completed.  The entire
procedure requires $d_G = \lceil \log_2(\ell_G) \rceil$ such calls of
{\sc DistributedMPM2}.  Therefore the total waiting times of {\sc
  D-IterSquare}, as a function of $n, \ell_G, p$ is
\begin{align*}
T_{D-IterSquare}(n, d_G, p) =& d_G p \left[\frac{1}{2}\log(p) (\kappa_L + \kappa_T (n/p) (n/\sqrt{p})) + 2\sqrt{p}\kappa_S +  \kappa_C n^2(2(n/p)+1)/p\right]\\
&= \frac{p \log (p)d_G }{2} \kappa_L + 2 d_G p^{3/2} \kappa_S + \frac{n^2 d_G}{\sqrt{p}} \kappa_T +  \frac{d_G n^2(2n + p)}{p}\kappa_C 
\end{align*}
Compare to distributed Floyd-Warshall:

\begin{algorithm}[H]
\caption{Distributed Floyd-Warshall}
\begin{algorithmic}
\Function{D-FloydWarshall}{Adjacency matrix $A$}
  \State $S \leftarrow A$
  \For{$k = 1,\hdots, n $}
    \State Let $S_{\cdot, k}$ denote the $k$th column and $S_{k, \cdot}$ denote the $k$th row of $S$
    \State Set $S' \leftarrow \text{\sc DistributedMPM}(S_{\cdot, k}, S_{k, \cdot})$
    \State $S \leftarrow \text{\sc DistributedMin}(S, S')$
  \EndFor
  \State Return $S$
\EndFunction
\end{algorithmic}
\end{algorithm}

Distributed Floyd-Warshall involves an alternating sequence of $n$ calls each to {\sc DistributedMPM} and {\sc DistributedMin}
where each call must be completed in sequence.  Therefore the total waiting time is
\[
T_{D-FloydWarshall}(n, d_G, p) =  \frac{n \log(p)}{2}\kappa_L + 2n\kappa_S +  \frac{n^3\log(p)}{\sqrt{p}}\kappa_T + \frac{3n^3}{p}\kappa_C
\]

It is then evident that for small log-diamterers $d_G$, large
latencies $\kappa_L$ and sufficiently large $p$, that
$T_{D-FloydWarshall}$ can exceed $T_{D-IterSquare}$.  Note that this
depends on knowing a good upper bound for $d_G$, which is possible in
many applications.

\section{Block Floyd-Warshall}

Note that both Floyd-Warshall and Iterative Squaring may be viewed as
special cases of a more general algorithm, \emph{Block Floyd-Warshall}
with parameters $K$ and $L$.

The parameter $K$ determines the number of both horizontal and
vertical \emph{blocks} to split $A$.  When $K = n$, each block is a
single row or column of $A$, as seen in the Floyd-Warshall update
step.  When $K = 1$, each block is the matrix $A$ itself, as seen in
iterative squaring.  The parameter $L$ determines the number of
\emph{outer loops}.  Floyd-Warshall has no outer loop, so $L=1$.
Iterative squaring has $L = d_G$ outer loops.

\begin{algorithm}[H]
\caption{Block Floyd-Warshall}
\begin{algorithmic}
\Function{BlockFloydWarshall}{Adjacency matrix $A$}
  \State Let $M = n/k$.
  \State $S \leftarrow A$
  \For{$l = 1,\hdots, L $}
    \For{$k = 1,\hdots, M$}
      \State Let $S = [S_1,\hdots, S_M]$, and $S^T = [s_1^T,\hdots, s_M^T]$
      \State $S \leftarrow \min(S,  S_k \otimes s_k)$
    \EndFor
  \EndFor
  \State Return $S$
\EndFunction
\end{algorithmic}
\end{algorithm}

For any given parameter $K$, Block Floyd-Warshall computes the APSP
matrix given $L \geq n/K$, as stated in the following theorem.


\subsection{Probabilistic Guarantees}

\subsection{Distributed Block Floyd-Warshall}

\bibliographystyle{abbrv}
\bibliography{isomap}



\end{document}
